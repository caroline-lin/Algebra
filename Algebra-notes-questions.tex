         
\documentclass[1    0pt, answers]{exam} \renewcommand{\baselinestretch}{1.05}
\usepackage{amsmath,amsthm,verbatim,amssymb,amsfonts,amscd, graphicx}
\usepackage{graphics}
\usepackage{paralist}

\usepackage{afterpage}
\usepackage{caption}

\usepackage{amstext} % for \text macro
\usepackage{array}   % for \newcolumntype macro


\usepackage{tikz}
\usepackage{fancybox}

\usepackage{clrscode3e}
\usepackage{etoolbox,refcount}
\usepackage{multicol}

\newcolumntype{L}{>{$}l<{$}} % math-mode version of "l" column type

\newcounter{countitems}
\newcounter{nextitemizecount}
\newcommand{\setupcountitems}{%
  \stepcounter{nextitemizecount}%
  \setcounter{countitems}{0}%
  \preto\item{\stepcounter{countitems}}%
}
\makeatletter
\newcommand{\computecountitems}{%
  \edef\@currentlabel{\number\c@countitems}%
  \label{countitems@\number\numexpr\value{nextitemizecount}-1\relax}%
}
\newcommand{\nextitemizecount}{%
  \getrefnumber{countitems@\number\c@nextitemizecount}%
}
\newcommand{\previtemizecount}{%
  \getrefnumber{countitems@\number\numexpr\value{nextitemizecount}-1\relax}%
}
\makeatother    
\newenvironment{AutoMultiColItemize}{%
\ifnumcomp{\nextitemizecount}{>}{3}{\begin{multicols}{2}}{}%
\setupcountitems\begin{itemize}}%
{\end{itemize}%
\unskip\computecountitems\ifnumcomp{\previtemizecount}{>}{3}{\end{multicols}}{}}


\topmargin0.0cm
\headheight0.0cm
\headsep0.0cm
\oddsidemargin0.0cm
\textheight23.0cm
\textwidth16.5cm
\footskip1.0cm
\theoremstyle{plain}
\newtheorem{theorem}{Theorem}
\newtheorem{corollary}{Corollary}
\newtheorem{lemma}{Lemma}
\newtheorem{proposition}{Proposition}
\newtheorem*{surfacecor}{Corollary 1}
\newtheorem{conjecture}{Conjecture}  
\theoremstyle{definition}
\newtheorem{definition}{Definition}

\begin{document}
\unframedsolutions

\section{Weeks 1 and 2}

\begin{questions}

%==1.1==
\begin{itemize}
    \item $G$ is a group and satisfies the group axioms:
    
    \begin{itemize}
        \item There is an identity;
        \item All elements have an inverse;
        \item The left and right inverses of an element must be equal.
        \item The group operation is associative.
    \end{itemize}
    
    \item The cancellation laws hold in $G$ on both sides (for noncommutative groups, too):
    \begin{itemize}
        \item $ab = ac$ implies that $b = c$;
        \item $ba = ca$ implies that $b = c$;
    \end{itemize}
    \item The \emph{order} of an element is the smallest power we can take before it becomes the identity.
    \begin{itemize}
        \item The \emph{only} element which can have order 1 in any group is the identity. (Because $x^1 = x$. Oops!)
        \item it's common for elements to have infinite order.
        \item The unit circle under the group operation of complex multiplication (rotation) has elements with finite order: all the roots of unity. For example, $e^{i*\pi} = -1$ has order 2, since $(-1)^2 = 1$.)
    \end{itemize}
    \item If $G$ has finitely many elements (read: "If $G$ is a \emph{finite group}) then we can explicitly list the elements $g_ig_j$ in an $n$ by $n$ matrix (called: the \emph{multiplication table}).
    \item Example: We are used to seeing multiplication tables from grade school, up to 10 by 10 or 12 by 12. Since integer multiplication is a commutative group, the multiplication table is symmetric.
\end{itemize}

\begin{theorem}[Proposition]
If $G$ is a group with group operation $\star$, then
\begin{enumerate}
    \item The identity is unique
    \item The inverse is unique for each $a$ in $G$
    \item The inverse is idempotent (read: $(a^{-1})^{-1} = a$)
    \item The inverse of a multiplication is the multiplication of the inverses, reversed ( $(a \star b)^{-1} = (b^{-1}) \star (a^{-1})$)
    \item The group operation is associative \emph{in general} (read: on sequences of operations $a_1 \star \ldots \star a_n$).
\end{enumerate}
\end{theorem}
\begin{proof} (In the text)
(3): Since $v \star v^{-1} = v^{-1} \star v = 1$, an inverse of $v^{-1}$ is $v$. From (2), inverses are unique; so $v$ is \emph{the} inverse of $v^{-1}$.
(4): Compute it. $(a \star b) \star (b^{-1} \star a^{-1}) = a \star (b \star b^{-1}) \star a = a \star (1) \star a^{-1} = a \star a^{-1} = (1)$. The rest comes from uniqueness of inverse.
(5): Gross induction proof.
\end{proof}



\question (20) Show that for $x \in G$, $x^{-1}$ and $x$ have the same order.
\begin{solution}
% Hussain's solution
Let $n$ be the order of $x$.
\[
(x^n)^{-1}\\
= 1\\
= (x^{-1})^n\\
= 1
\]
\end{solution}

\begin{solution}
% Caroline's solution here
Use the group `axioms'.
If $x^n = 1$, then notice $1 = 1^{-1} = (x^n)^{-1} = (x^{-1})^n$ by (4) from the proposition. We're done.
\end{solution}

\question (25) Prove that if $x^2 = 1$ for all $x$ in $G$, then $G$ is abelian.
\begin{solution}
% Hussain's solution
Let $y \in G$.

By definition, $(yx)(xy)^2 = (yx)$.

By the transitivity property: $(yx)(xy)(xy) = y(xx)y(xy) = (yy) xy = xy = yx$.

Thus, $G$ is commutative.
\end{solution}

\begin{solution}
% Caroline's solution here
Show that the elements commute.
If $a^2 = 1$ we can show $a = a^{-1}$ (use the cancellation property).
then $ab = a^{-1}b^{-1} = (ba)^{-1} = ba$, so $G$ is abelian. 
\end{solution}

%==1.2==
Notes.
\begin{itemize}
    \item Dihedral groups: Are \emph{literally} the rigid motions of regular $n$-gons in 2D Euclidean space. They are denoted $D_{2n}$ (because an $n$-gon has $2n$ rigid motions.)
    \begin{itemize}
    \item The dihedral group elements act on their respective $n$-gons. We'll define this more formally in a later section.
    \item How many elements are in the dihedral group for an $n$-gon relative to $n$? The answer is $2n$. There are $n$ vertices, and the rigid body motion moves a vertex to another vertex after the transformation. Consider vertex 1. There are $n$ distinct  transformations that move vertex 1 to vertex $i$ ($i$ can also equal 1. This is the identity rotation).
    \item Next, we look at the vertices adjacent to vertex 1. Rigid motions have to preserve the vertices which are adjacent to vertex 1, so after any transformation, we either have 0 - 1 - 2 or 2 - 1 -0 as a sequence of adjacent vertices.
    \item Depending on which is chosen, we know the entire vertex layout. For example, on a square, setting the vertex sequence to be 2 - 1 - 0 means that vertex 3 must be ``on the left'' of 2 and ``on the right'' of 3.
    \item For each rotation, there are two vertex sequences (these correspond to the flips). Hence the dihedral group $D_{2n}$ on an $n-$gon has exactly $2n$ (unique) elements.
    \end{itemize}
    
    \item Denote by $r$ a clockwise rotation which moves vertex $i$ to vertex $i + 1$ mod $n$, and let $s$ be a flip through the axis given by vertex 0 and the origin. Then:
    \begin{itemize}
        \item The elements $1, \ldots, r^{n-1}$ are all distinct. $r^n = 1$, so the order of $r$ is $n$.
        \item The order of $s$ is 2.
        \item It's impossible to have $s = r^i$ (read: flips cannot be made from rotations)
        \item It's impossible to have (non-trivially) $sr^i = sr^j$. Non-trivial means $i \neq j \mod n$. (read: flips are bijections from the set of cw rotations to the set of ccw rotations.)
        \item $rs = sr^-1$. (You.. can literally think of this following from the restatement, where rs = r(s acting on 1) = r, since the flipped identity is also the identity.)
        \item ``If you wanna commute powers of $r$...'' $r^is = sr^{-i}$. (Use induction on $i$...)
    \end{itemize}
    \item Generators. (Analogy to basis). A set of group elements $S$ is a generator of its group $G$ if every element in $G$ can be expressed as a finite product of elements in $S$ (and their inverses. Note $S$ need not be a subgroup, just a subset.) We ,may also say $S$ \emph{generates} $G$.
    \item Relations. Are satisfied by the group generators, e.g. $r^n = 1$ in the dihedral group. Relations are used for simplifying expressions, and can help us determine if two group elements are equal.
    \item Presentations. Are a set $S$ of group elements with a (finite) set of relations $\mathcal{R} = R_1, \ldots, R_m$. The relations must be \emph{generating relations} in the sense that all other relations on the group must be derivable from the set.
    \item A presentation is denoted by $G = \langle S | \mathcal{R} \rangle$. A presentation of the dihedral group $D_{2n}$ is $\langle r, s | r^n = s^2 = 1, rs = sr^{-1} \rangle$. 
    \item the book talks about how under a general presentation of a group we find it difficult to 1) resolve whether two elements are equal or 2) resolve the group determined by a particular presentation. This should be obvious to anyone who's done a bit of formal CS or logic.
\end{itemize}

\question If $n$ is odd, and $n \geq 3$, show that the identity is the only element of $D_{2n}$ which commutes with all elements of $D_{2n}$.
\begin{solution}
% Hussain's solution
Let $0 < i < n$.

Then either elements of $G$ are of the form $r^i$, xor $r^i s$.

In the first case:
$r^i s = s r^{-i} = s r^{n-i}$,
and the commutation is $s r^i$.
These are only equal when $n-i = i \leftrightarrow n = 2i$,
which contradicts the assumption that $n$ is odd.

Thus, $r^i$ does not commute with every element of $G$, because it does not commute with $s$.

In the second case:
$(r^i s) s = r^i s^2 = r^i$,
and the commutation is $s (r^i s) = s (s r^{-i}) = s^2 r^{-i} = r^{n-i}$.
Similarly, these are only equal when $i = n-i \leftrightarrow n = 2i$,
which contradicts $n$ being odd.

Thus, $s r^i$ does not commute with all members of $G$, because it does not commute with $s$.

By exhaustion, no member of $G$ except the identity commutes with all other members of $G$, because they all do not commute with $s$.

The geometric intuition for this is as follows:
for a flip and a rotation to commute, the flip has to preserve the location of the vertex v1 that the rotation takes v0 to.
So: the rotation takes v0 to v1, then the flip preserves v1, or the flip preserves v1, then the rotation takes v0 to v1.

Flips only preserve vertices that are opposite to v0 (on the same reflection axis).
Only dihedral groups of even order have reflection axes that connect pairs.
Thus, flips and rotations don't commute in dihedral groups of odd order.
\end{solution}

%Caroline's clunky solution
\begin{solution}
None of the rotations $g = r^i$ commute with $s$ ($ri s \neq s ri$ for all $i$), because $r^i s = s r^{-i}$ implies $r^i = r^{-i}$ implies $-i = i$ mod $n$. Then $i = -i$ mod $n$ implies $n$ divides $2i$. If $i$ is even this can only happen if $i = 0$ (so $r^i$ is the identity); if $i$ is odd then $i$ has $n$ as a factor, but since $r^n = 1$, we can rewrite the first eqn so $n$ is not a factor. Repeat this second argument (which diminishes $i$ each time, since $n \neq 1$) until we are forced to conclude that $i = 0$.

None of the flipped rotations $g = sr^i$ commute with $s$ either, because $s (s r^i) = s (r^i s)$ implies $r^i = s (s r^{-i})$ implies $r^i = r^{-i}$ again.
\end{solution}

%==1.3==
Notes.
\begin{itemize}
    \item Denote by $S_{\Omega}$ the set of all permutations (read: bijections) of the set $\Omega$. This is a group with composition as the group operation. Call it the \emph{symmetric group} on $\Omega$.
    \item For finite sets $\Omega$ with size $n$, we say $S_{\Omega}$ has degree $n$, and abuse notation to write $S_n = S_{\Omega}$.
    \item The order (read: size) of $S_n$ is $n$ factorial. (Proof: Stats 257.)
    \item Cycles: Denoted by $(a_1, \ldots, a_m)$, where the $a_i$ are elements of $\Omega$. Interpretation:$\sigma$ sends $a_i$ to $a_{i + 1}$ (mod $m$). All permutations are ``decompositions'' of cycles (trivially).
    \item The Cycle Decomposition algorithm: walk until you hit a wall, then start from somewhere else, until you've gotten everything. 
    \item More seriously: To decompose $\sigma$ into a set of cycles, pick arbitrary $a$ in $\Omega$, then apply $\sigma$ to it until you get the smallest $i$ such that $\sigma^i(a) = a$. That's a cycle. Pick some other element not yet found, and repeat the last step with that new element. Repeat the third step until all elements have been assigned to some cycle.
    \item You can compute products of permutations in a dumb and easy way using decompositions. Also, cycle decompositions are unique.
    \item example: $(1,2) \circ (1,3) = (1, 3, 2)$ and $(1, 3) \circ (1, 2) = (1, 2, 3)$.
    \item By the way, the last example shows that $S_n$ is non-abelian. However, \emph{disjoint} cycles commute.
    \item Useful fact: The order of a permutation is the \emph{lowest common multiple} of the lengths of all of its cycles. (Thus every element in $S_n$ must have finite order. Neat!)
\end{itemize}

\question (optional). Let $\sigma$ be the permutation
\[ 1 \mapsto 3, 2 \mapsto 4, 3 \mapsto 5, 4 \mapsto 2, 5 \mapsto 1 \]
and let $\tau$ be the permutation
\[ 1 \mapsto 5, 2 \mapsto 3, 3 \mapsto 3, 4 \mapsto 4, 5 \mapsto 1. \]
Find the cycle decompositions of the following permutations:
\begin{align*}
\sigma \\
\tau \\
\sigma^2 \\
\sigma \tau \\
\tau \sigma \\
\tau^2 \sigma
\end{align*}

\begin{solution}
% Hussain's solution
% solved with helper python script
$\tau$ is not a bijection! (nothing maps to $2$.)
\begin{verbatim}
>>> de.decompose_cycles(s)
[[5, 1, 3], [4, 2]]
>>> de.decompose_cycles(de.compose(s, s))
[[5, 3, 1], [4], [2]]
\end{verbatim}
\end{solution}


\begin{solution}
% Caroline's solution here... later
\end{solution}

%==1.4==
Notes
\begin{itemize}
    \item The General Linear group: It's the set of all nonsingular matrices over a field $F$ (Why this matters: these are the linear transformations of a vector space $V$ which take bases onto bases. They can also be thought of as the isomorphisms of the space.)
    \item It is denoted by $GL_n(V)$, where $V$ is a vector space.
    \item It is a group! Under composition. We can check this with the nifty formula $\det(A \dot B) = \det(A) \det(B)$. The identity is the identity transformation, and the inverse on an element is just its inverse matrix.
    \item Fields: Fields are endowed with two group operations, $+$ and $\cdot$. The distributive identity $a(b + c) = ab + ac$ holds. We also require the additive, multiplicative groups to be abelian (so stuff commutes).
    \item Fact no. 1. If $F$ is a field of finite order (read: size), then the size of $F$ is equal to $p^k$, where $p$ is prime, and $k$ is an integer.
    \item Fact no. 2. Also, if $F$ is a field of finite order $q$, then the general linear group over $F$ has size equal to $(q^n - 1)(q^n - q)(q^n - q^2)\ldots(q^n - q^{n-1})$. Neat!
\end{itemize}

\question (2) Write out all the elements of $GL_2(\mathbb{F}_2)$, and compute the order of each element.

(My translation: write all 2x2 invertible matrices over the field with only two elements. Compute order.)

\begin{solution}
\[
\begin{bmatrix}
1 & 0 \\
0 & 1
\end{bmatrix} \] 
This has det = 1 - 0 = 1, and order 1 -- it's the identity!
\[
\begin{bmatrix}
0 & 1 \\
1 & 0
\end{bmatrix} \]
This has det = 0 -1 = -1 = 1, and order 2 (e1 gets sent to e2, and vice versa, so just take it twice).
Since $0 \times 1 = 0 \times 0 = 0$, and $1 - 1 = 1 + 1 = 0$, there aren't any other matrices.
\end{solution}

%=="this handout"==
\question For every integer $n$, find the number of elements of order $n$ in $S_5$.

%(What's S_5?) -- Ans: Oh, it's the symmetric (permutation) group on sets of 5 elements.
\begin{solution}
% Hussain's solution:
Every element can be decomposed into disjoint cycles.
The order of an element is the lowest common multiple of the lengths of these cycles.
The highest this can be is $6$ for $2 \times 3$, so there are no elements of order $n$ for $n > 6$.

For order $5$, there are only cycles of length $5$
with quantity $4! = 24$
(obtained by fixing a reference first-element and enumerating possible paths).

For order $4$, there are only cycles of length $4$ and $1$
with quantity $5 \times 3! = 30$.

For order $6$, there are only cycles of length $2$ and $3$,
with quantity $5 \times 2 \times 2 = 20$.

For order $3$, there are only cycles with lengths of $3$, $1$, and $1$,
with quantity $(5 \times 2) \times 2 = 20$.

For order $2$, there are cycles with lengths of $2$, $1$, $1$, and $1$,
with quantity $5 \times 2 = 10$,
and with paths $2$, $2$, and $1$,
with quantity $5 \times 3 = 15$.

For order $1$, there is only trivially the identity bijection.

This totals up to $120 = 5!$, which is the order of the group.
\end{solution}

%Caroline-soln
\begin{solution}
Using cyclic decomposition notation for permutations.
1: It's the identity permutation.
2: All permutations of the form (a b)(c d)(e) or (a b)(c)(d)(e)
3: Just (a b c)(d)(e).
4: Just (a b c d)(e).
5: (a b c d e).
That was easy.
\end{solution}

%==1.5==
Notes
\begin{itemize}
\item The! Quaternion! Group!! Is given by $Q_8 = \{ 1, -1, i, -i, j, -j, k, -k \}$ (these are the unit quaternions.)
\item The group operation is quaternion multiplication. The quaternion multiplication rules are:
\begin{itemize}
    \item $(-1)^2 = 1, i^2 = j^2 = k^2 = -1$;
    \item $ij = k, ji = -k$;
    \item $jk = i, kj - -i$;
    \item $ik =j, ki = -j$;
    \item $1 \dot a = a$, for all $a \in Q_8$.
\end{itemize}
\item Clearly, $Q_8$ is not abelian.
\end{itemize}

\question (2) Write out the group tables for $S_3, D_8 and Q_8$.
% S_3 -> symmetry (permutation) group on three els
% D_8 -> dihedral group on a square
% Q_8 -> quaternion group
\begin{solution}
% Hussain's solution
% Hussain hates tabulating things go away.
\end{solution}

\begin{solution}
Table is always of the form (col) compose (row) -- this is consistent with the book's notation
`` The (i, j) entry contains $g_i g_j$''.
\renewcommand\arraystretch{1.3}
\setlength\doublerulesep{0pt}
For $S_3$ (6 elements), the diagonal eles in the first four should be the identity, whoops\\
\pagebreak

\begin{tabular}{r||c|c|c|c|c|c||}
. &         (1)(2)(3) & (1, 2)(3) & (1)(2, 3) & (2)(1, 3) & (1, 2, 3) & (3, 2, 1)  \\
\hline\hline
(1)(2)(3) & (1)(2)(3) & (1, 2)(3) & (1)(2, 3) & (2)(1, 3) & (1, 2, 3) & (3, 2, 1) \\ 
\hline
(1, 2)(3) & (1, 2)(3) & (1)(2)(3) & (1, 2, 3) & (1, 2, 3) & (1, 3)(2) & (1)(2, 3)\\ 
\hline
(1)(2, 3) & (1)(2, 3) & (1, 2, 3) & (1)(2)(3) & (3, 2, 1) & (1, 2)(3) & (1, 3)(2)\\ 
\hline
(2)(1, 3) & (2)(1, 3) & (3, 2, 1) & (1, 2, 3) & (1)(2)(3) & (2, 3)(1) & (3)(1, 2)\\ 
\hline
(1, 2, 3) & (1, 2, 3) & (1)(2, 3) & (1, 3)(2) & (1, 2)(3) & (1, 2, 3) & (1)(2)(3) \\
\hline
(3, 2, 1) & (3, 2, 1) & (3)(1, 2) & (3)(1, 2) & (3, 2)(1) & (1)(2)(3) & (3, 2, 1)\\
\hline
\end{tabular}\\
Some rough work for entry (col, row)
$(2,3): (1 -> 1 -> 2, 2 -> 3 -> 3, 3 -> 2 -> 1)$
$(2,5): (1 -> 2 -> 1, 2 -> 3 -> 3, 3 -> 1 -> 2)$
$(3,2): (1 -> 2 -> 2, 2 -> 1 -> 3, 3 -> 3 -> 1)$

For $D_8$ % T_T check this
\begin{tabular}{L||L|L|L|L|L|L|L|L||}
.    &   1  & r    & r^2  & r^3  & s    & sr   & sr^2 & sr^3  \\
\hline\hline
1    &   1  & r    & r^2  & r^3  & s    & sr   & sr^2 & sr^3 \\
r    &   r  & r^2  & r^3  & 1    & sr   & sr^2 & sr^3 & s    \\
r^2  & r^2  & r^3  & 1    & r    & sr^2 & sr^3 & s    & sr   \\
r^3  & r^3  & 1    & r    & r^2  & sr^3 & s    & sr   & sr^2 \\
s    &   s  & sr^3 & sr^2 & sr   & 1    & r^3  & r^2  & r^1  \\
sr   &  sr  & 1    & sr^3 & sr^2 & r    & 1    & r^3  & r^2  \\
sr^2 & sr^2 & sr   & s    & sr^3 & r^2  & r    & 1    & r^3  \\
sr^3 & sr^3 & sr^2 & sr   & s    & r^3  & r^2  & r    & 1\\
\hline
\end{tabular}\\

For $Q_8$
% Terrible.
\begin{tabular}{L||L|L|L|L|L|L|L|L||}
.  & 1  & i  & j  & k  & -1 & -i & -j & -k  \\
\hline\hline
1  & 1  & i  & j  & k  & -1 & -i & -j & -k \\
i  & i  & -1 & -k & -j & -i & 1  & k  & j  \\
j  & j  & k  & -1 & -i & -j & -k & 1  & i  \\
k  & k  & j  & i  & -1 & -k & -j & -i & 1  \\
-1 & -1 & -i & -j & -k & 1  & i  & j  & k  \\
-i & -i & 1  & k  & j  & i  & -1 & -k & -j \\
-j & -j & -k & 1  & i  & j  & k  & -1 & -i \\
-k & -k & -j & -i & 1  & k  & j  & i  & -1 \\
\hline
\end{tabular}\\
\end{solution}

%==1.6== Homomorphisms and Isomorphisms
Notes 
\begin{itemize}
\item Relating two groups: We have some facts and theorems about group $A$ and we would like to apply them in some way to group $B$. We see that $A$ and $B$ are `very similar' in some way -- multiplication on elements of $A$ somehow behaves similarly on other elements of $B$. How can we relate the two groups?
\item Homomorphisms: A homomorphism is an operation $\phi$ which relates the group operations of two \emph{separate} groups. Formally, \emph{the image of group elements under a homomorphism behave well with respect to the group operation in the new group}.
\item Notation: a homomorphism is a transformation $\phi : G_1 \to G_2$ such that for any two elements $x, y$ in $G_1$, $\phi(x \star y) = \phi(x) \diamond \phi(y)$.
\item In other words, a well-defined homomorphism allows us to move \emph{between groups}, using whichever group operation we prefer, without a loss of information.
\item Example: Let $H$ be a subgroup of $G$. Then the inclusion map $ i : H \to G$ is a homomorphism of $H$ to $G$.
\item Isomorphism: there is another desirable property to have when discussing two groups: it is bijectivity. We say that a homomorphism  of $G$ and $H$ which is also bijective is an \emph{isomorphism} of $G$ and $H$. Groups for which an isomorphism exists are said to be \emph{isomorphic} to each other; they can also have the same \emph{isomorphism type}.
\item Example: The exponential map taking $\mathbb{R} \to \mathbb{R}^{+}$, $x \mapsto e^{x}$, takes the additive reals to the multiplicative reals (read: an isomorphism of the reals. q: in what world are we restricted from multiplication, but are allowed to exponentiate/ is this ever useful?)
\item Example: All of the symmetric groups on finite sets of order $n$ are isomorphic. Reason: Define $\phi$ to be any bijection between the two sets $\Omega$ and $\Delta$ (we know such a bijection $\theta$ exists because by definition of finite of order $n$, there are bijections from both sets to the set $\{1, \ldots, n \} \subseteq \mathbb{N}$). 
\item Let $\sigma$ be a permutation on $\Delta$ and $\phi$ be such that $\sigma(x) = y \implies \phi(\sigma)(\theta(x)) = \theta(y)$ in $\Omega$. (``Pick the bijective $\phi$ on permutations which works after fixing the bijective $\theta$ on set elements.'')
\item Conversely, it can be shown that two isomorphic symmetric groups must have underlying sets of the same order (even for infinite sets $\Omega$ and $\Delta$). (The finite case is very easy, just assert that $n! = m!$ implies $n = m$ % sorry for inconsistent notation, the ideas are all really simple
\end{itemize}

\begin{itemize}
\item Classification theorems: properties categorising a set of groups related under isomorphism, e.g. any non-abelian group of order 6 is isomorphism to $S_3$ (the only other order 6 group is $\frac{\mathbb{Z}}{6 \mathbb{Z}}$) % huh
\item (Condition for isomorphisms of generated groups.) Suppose we have $m$ elements which generate the group $G$, and $m$ elements which belong to the group $H$. If a bijection exists where $\psi(g_i) = h_i$ \emph{preserves the relations imposed on the $g_i$ in the presentation of $G$}, then there is in fact a unique \emph{homomorphism} from $G$ to $H$ (probably as an extension of $\psi$, but we'll see later)
\item Note: if $|G| = |H|$ then this homomorphism becomes an isomorphism.
\item The corresponding statement for vector spaces: Two $n$-dimensional vector spaces are isomorphic by (some invertible linear transformation of the basis in $V$ to the basis in $W$).
\item (Then some examples)
\end{itemize}

\begin{theorem}
If $G$ and $H$ are isomorphic groups, then
\begin{itemize}
\item $|G| = |H|$
\item $G$ abelian iff $H$ abelian
\item For all $x \in G$, $|x| = |\phi(x)|$ (the orders of $x$ and $\phi(x)$ must be equal)
\end{itemize}
\end{theorem}

\question (17) Let $G$ be any group. Prove that the map from $G$ to itself defined by $g \mapsto g^2$ is a homomorphism if and only if $G$ is abelian.
\begin{solution}
% Hussain's solution
\[
(xy)^2 = x^2y^2\\
x (y xy) = x (x y^2)\\
(y x) y = (x y) y\\
y x = x y\\
\]
\end{solution}

%Caroline's soln, yay
\begin{solution}
Use the homomorphism condition $\phi(xy) = \phi(x)\phi(y)$. In this case it reduces to $x^2 y^2 = (xy)^2 = (xy)(xy)$ which is equivalent to $xy = yx$ after left-multiplying by $x^{-1}$ and right-multiplying by $y^{-1}$ (cancellation property).
\end{solution}

%==1.7== Group Actions -- see supp notes below this section for more defns

% Any group which supports an action on $A$ is in fact homomorphic to the group of permutations on A ``the symmetric group''
% ie. ``every element in $G$ acts as a permuation on $A$ in a manner consistent with the group operation on $G$''

\begin{theorem}
Fix $g \in G$. If $G$ is a group which acts on $A$, define the map $\sigma_g : A \to A$ to be the evaluator $g \cdot a$. Then
\begin{itemize}
    \item sigma is actually a permutation of $A$
    \item the map $g \mapsto sigma_g$ is a homomorphism.
\end{itemize}
\end{theorem}
\begin{proof}
Property (1) holds because of the `associativity' property of actions. For any $\sigma_g$, its inverse is $\sigma_{g^{-1}}$:
\begin{align*}
( \sigma_g \circle \sigma_{g^{-1}})(a) = g \circle (g^{-1}(a)) = (g \circle g^{-1})(a) = (I)a = a.
\end{align*}
The first equality was from defn of composition. This is actually sufficient, because remember, `permutation' means `bijection'. The sigmas map $A$ back to itself and were just shown to have (two-sided) inverses; thus they are bijective (see also prop'n 1 of section 0.1).
Property (2): Using the above definition of the homomorphism $\phi$, we compute, for all $a \in A$ and for aribtrary group elements of $G$, $g_1$ and $g_2$:
\begin{align*}
\phi(g_1 g_2)(a) & = \sigma_{g_1 g_2}(a) = (g_1 g_2) \cdot a  = g_1 \cdot ( g_2 \cdot a) \\
& = \sigma_{g_1} ( \sigma_{g_2} (a)) = (\phi(g_1) \circle \phi(g_2))(a)
\end{align*}
\end{proof}

\begin{itemize}
\item (then a bunch of examples)
\end{itemize}

\question (17) Let $G$ be a group, and let $G$ act on itself by left conjugation, so that each $g \in G$ maps $G$ to $G$ by $$ x \mapsto g x g^{-1}.$$
For fixed $g \in G$, prove that conjugation by $g$ is an isomorphism from $G$ onto itself (i.e., is an automorphism of $G$ -- see exc. 20 in section 6). Deduce that $x$ and $g x g^{-1}$ have the same order for all $x$ in $G$, and that for any subset $A$ of $G$, $|A| = |g A g^{-1}|$ ( here $g A g^{-1} = \{g a g^{-1} | a \in A \}$.)

\begin{solution}
% Hussain's solution
Homomorphism:
\[
(g (x) g^{-1}) (g (y) g^{-1})\\
= g (x) g^{-1} g (y) g^{-1}\\
= g (x) (y) g^{-1}\\
= g (xy) g^{-1}
\]

injectivity:
\[
g x g^{-1} = g y g^{-1}\\
g x = g y\\
x = y
\]

Surjectivity:
Define the inverse map as $x \mapsto g^{-1} x g$.
This is also homomorphic and injective, per above.

order:
These just follow from the fact that it is a group isomorphism.
\end{solution}

%==Supplementary Notes: Counting, Group Actions, and the Orbit-Stabilizer lemma==
\begin{itemize}
\item Many groups are actions on sets. We say $G$ acts on $A$ (on the left) if we can build an evaluation function $\alpha: G \times A \to A$ taking $(g, a) \mapsto g \cdot a$. The evaluation satisfies the following:
\begin{itemize}
    \item `associativity' holds: $g \cdot (h \cdot a) = (g \cdot h)(a)$
    \item The identity must have the expected result: $\alpha(1, a) = a$. Note that the first property does \emph{not} imply the second.
\end{itemize}
\item Fixed set: The fixed set of an action is the set of objects left invariant by that action (formally: for $g \in G$, $\text{Fix}(g) \equiv \{ a \in A | g \cdot a = a \}$)
\item Stabilizer: Conversely, the set of all actions which \emph{fix} $a$ is called the \emph{stabilizer} of $a$.
\item Orbit: The image of $a$ under all group actions is the \emph{orbit} of $a$, denoted by $\Omega_a$.
\end{itemize}

% Q2 - Classify the actions
\question Which of the following are proper actions?
\begin{parts}
\part The dihedral group acting on the vertices of a regular $n$-gon.
\part The dihedral group acting on the \emph{diagonals} of a regular $n$-gon.
\part An arbitrary group acting on \emph{itself}, using the group operation as the evaluation function: $\alpha(g, a) = a \cdot g$.
\part An arbitrary group acting on itself, using $\alpha(g,a) = g a g^{-1}$ as the evaluation function.
\end{parts}

\begin{solution}
% Hussain's solution
---dihedrals on vertices
Yes. The underlying permutation sets look like this:
flip:
for order $2n$: $n$ 2-cycles
for order $2n+1$: $n$ 2-cycles and $1$ 1-cycle
rotations by r:
for order $2n$ and $2n+1$: $n$ cycles of length $1 + (r^{-1} mod n)$

---dihedrals on diagonals
Yes, but the permutation sets are too complicated to list out.

---group right-multiplication
Not necessarily a homomorphism:
\[
a.(g.h)\\
= (a.h).g\\
= a.(h.g)\\
\leftrightarrow g.h = h.g
\]

Which is not true for non-abelian groups.

-group conjugation
Yes.
compatibility:
\[
(g.h) a (g.h)^{-1}\\
= g (h a h^{-1}) h
\]

And trivially for the identity ($g E g^{-1} = E$).
\end{solution}
% Educated guesses
\begin{solution}
Yes.
Yes! (If I compose two rigid motions and then apply them to the triangle, is it the same as applying each rigid motion to the triangle, successively?) 
Nope (has to be commutative lol -- for non-abelian groups associativity $(ab)(c) = (a)(bc)$ implies $c(ab) = bc(a)$. Take b to be the identity and you get commutivity.) But notice using $\alpha(g, a) = g \cdot a$ would actually work.
No -- $ab(c) = a(bc)$ implies $ab c (ab)^{-1} = a (bc) a^{-1}$ implies $c b^{-1} a^{-1} = a bc a^{-1}$ implies $c b^{-1} = abc$, which is pretty clearly not true in general.
\end{solution}

% extra questions from the supplementary notes (see 3blue1brown)
\question Say we want to count how many different necklaces we can build with 6 stones each, if we have stones of two different colours. Define a \emph{diagram} to be any way of colouring each of the six vertices of a hexagone with black or white. Notice that $|A| = 64$. (?) Show that $D_{12}$ acts on $A$, and that the number of orbits of this action is equal to the number of different necklaces.
Note: This shows that the problem of counting the number of orbits of an action is an interesting problem in combinatorics.

\begin{solution}
% hussain's attempt
Let $R$ be the following $6 \times 6$ matrix:
\[
\begin{matrix}
0 & \cdots & 1\\
1 & \ddots & 0\\
\vdots & \cdots & \vdots\\
\cdots & 1 & 0
\end{matrix}
\]
(which rotates vector components),

and $S$ be the folowing $6 \times 6$ matrix:
\[
\begin{matrix}
0 & \cdot & 1\\
& \vdot & \ddot & \vdot\\
1 & \cdot & 0
\end{matrix}
\]
(which flips vectors).

Represent a member $a \in A$ as a vector $v \in (Z/2Z)^6$, with $1$ for white, $0$ for black, the first component representing the top-most vertex on the hexagon, and proceding counter-clockwise.

Then, $S_12$ acts on $A$ as follows:
$$(r, v) \mapsto Rv$$,
$$(s, v) \mapsto Sv$$.

In particular, $R^{-1} = R^T$, $R^6 = I_6$ and $S^2 = I_6$, and we have:
$r.(r.a) = (r.r).a = R^2 v$,
$r.(s.a) = (s.r).a = (s.r^{-1}).a = S R^{-1} v$,
$(E).a = I_6 v = v = a$,
$s.(s.a) = (s.s)a = S^2 v = I_6 v = v$.

%rest about counting orbits

\end{solution}
\question Consider the following diagrams: ((some diagrams of 6-necklaces))
For each necklace, compute the size of its orbit, and the size of its stabilizer. Make a conjecture or formula that relates these two numbers for an arbitrary element in an arbitrary action. Prove it.
\end{questions}

% referenced by q19
\section{Weeks 3 and 4}
\begin{questions}
% Note -- section 0.2 was assigned as supplmentary reading this week and I may add notes here later

%==1.7== buh buh
\subsection{1.7}

% referenced by q19
\question (18) Let $H$ be a group acting on a set $A$. Prove that the relation $\sim$ on $A$ defined by \[ a \sim b \text{ if and only if } a = hb \text{ for some } h \in H \] is an equivalence relation. (For each $x \in A$ the equivalence class of $x$ under $\sim$ is called the \emph{orbit} of $x$ under the action of $H$. The orbits under the action of $H$ [artition the set $A$. 

% Caroline's
\begin{solution}
Aside -- I think this is just proving that right cosets of $H$ are well-defined equivalence classes given that $H$ is a subgroup? Anyways.
\begin{itemize}
\item Reflexivity. We can show that the identity belongs to any subgroup following the subgroup axioms. Hence for all $a \in G$, $a = (1)a$ implies that $a \sym a$.
\item Symmetry. Suppose $a \sym b$. Then $a = hb \implies b = h^{-1}a$ (by group action associativity + action of identity on $A$). Since subgroups are closed under inverses, $h^{-1} \in H$ and so $b \sym a$.
\item Transitivity. Suppose $a \sym b$ and $b \sym c$. Then $a = hb, b = h'c$ so $a = hh'c$ where $h, h'$ belong to $H$. By closure of subgroups under the group operation, $a \sym c$.
\end{itemize}
\end{solution}

\question (19) Let $H$ be a subgroup (see exc. 26 of section 1) of the finite group $G$ and let $H$ act on $G$ (here $A = G$) by left multiplication. Let $x \in G$ and let $\mathcal{O}$ be the orbit of $x$ under the action of $H$. Prove that the map \[ H \to \mathcal{O} \text{ defined by } h \mapsto h x \] is a bijection (hence all orbits have cardinality $|H|$). From this and the preceding exercise deduce \emph{Lagrange's Theorem}: \\
\emph{if $G$ is a finite group and $H$ is a subgroup of $G$, then $|H|$ divides $|G|$.}

\begin{solution}
(This is kind of `baby lagrange' since it works only for finite sets.)
That the map is surjective follows from the definition of the orbit of $x$. Why is it injective? Suppose $h, h' \in H$ and $hx = h'x$. Then by the cancellation property, $h = h'$. Hence, every distinct orbit has exactly $|H|$ elements.
By exercise 18 above, $H$ partitions $G$ into \emph{disjoint} sets (specifically, by the transitivity property of equivalence relations). Since every $x\in G$ belongs to some orbit, it follows that $|G|$ is an integer multiple of $|H|$, or equivalently, that $|H|$ divides $|G|$. (We cannot have fractional numbers of elements, nor can a single element belong to more than one $H$. Fooey.)
\end{solution}

%==2.1== (Subgroups)
\begin{itemize}
\item Subgroups of $G$. The set $H$ is a subgroup of $G$ if it is closed under the group operation, and closed under taking inverses. (formally: $x, y \in H$ implies $x^{-1} \in H$ and $x y \in H$). Subgroups $H$ of the group $G$ can be denoted by $H \leq G$.
\item Also, $H$ cannot be empty.
\item Check for a subgroup: Proposition. $H$ is a subgroup if
\begin{itemize}
    \item $H$ is not empty;
    \item for all $x, y \in H, xy^{-1} \in H$.
    \item Proof: (eh, later.)
\end{itemize}
\end{itemize}

\question (4) Give an explicit example of a group $G$ and an infinite subset $H$ of $G$ that is closed under the group operation, but is not a subgroup of $G$.

% Caroline's solution
\begin{solution}
Let $G$ be the integers with addition as the group operation, and let $H$ be the positive integers. The sum of any two positive integers is another positive integer, but no integer has an additive inverse. (We can throw in zero to catch the identity, but I don't think this was required.)
\end{solution}

%==2.2== (Optional for now) Centralizers, Normalizers, Stabilizers and Kernels
\begin{itemize}
\item Centralizer. Let $A$ be a subset of $G$. For some elements $g$ of the group, elements of $a$ will be invariant in the following way: $g a g^{-1} = a$. The \emph{centralizer} of $A$ in $G$ is the set of all elements $g$. We may denote it by $C_{G}(A) = \{ g \in G | \forall a \in A,  g a g^{-1} = a \}$.
\item The centralizer of any subset is a subgroup of $G$.
\item Proof: Let x,y be centralizers of $A$ in $G$. Then for $a \in A$, $(xy) a (xy)^{-1} = x(yay^{-1})x^{-1} = a$. Also, if $x$ is a centralizer of $A$, then so is $x^{-1}$, since $x a x^{-1} = a \implies x^{-1}x a x^{-1}x = x^{-1} a x \implies a = x^{-1} a x$. Finally, the identity belongs to the centralizer of any set.
\item Center. The \emph{center} of $G$ is the set of all elements which commute with $G$. It is denoted by $Z(G) = \{ g \in G | \forall x \in G, gx = xg \}$. It is a special case of the centralizer: $C_G (G) = \{ g \in G | \forall x \in G g^-1 x g = x \}$ Multiplying the equality on both sides by $g$, we obtain, equivalently, $\forall x \in G, x g = g x $, the definition of elements in the center.
\item Normalizer. The \emph{normalizer} of a set $A$ in $G$ is the set of elements $g$ which leave $A$ invariant \emph{as a whole}  under the action of $g$. Formally, denote this by $N_G (A) = \{ g \in G | g A g^{-1} = A \}$, where the set $g A g^{-1} = \{ g a g^{-1} | a \in A \}$. The normalizer of $A$ contains the centralizer of $A$ as a (strict) subset in general. The normalizer is also a subgroup of $G$.
\item Examples:
\begin{itemize}
    \item Abelian groups have $G$ as their center ($Z(G)$). In fact, $C_G (A) = G$ for any subset $A$, implying that the normalizer of $A$ is also $G$ for all $A$.
    \item The (non-commutative) dihedral group $D_8$ has an commutative (abelian) subgroup $A = \{ 1, r, r^2, r^3 \}$:
    \item The centralizer of $A$ is $A$;
    \item The normalizer of $A$ is $G$ (flipping an n-gon twice reverts it back to the original rotation-orientation);
    \item The center of $D_8$ is $\{1, r^2 \}$.
    \item Let $G$ be the symmetric group (permutation group) on 3-sets $S_3$, and $A$ be the subgroup $\{ (1), (1 2) \}$:
    \item The center and normalizer of $A$ are both $A$ (One proof is by Lagrange's Theorem);
    \item The center of $G$ is the identity alone.
\end{itemize}
\item Stabilizers and Kernels: Literally the same thing, but replace $A$ with some arbitrary set $S$, and let $G$ act on $S$.
\item Preamble: The proof of subgroupiness of the centralizer and normalizer relies on theorems about group actions. This indicates that $G$ is somehow determined by its action on its subsets.
\item Stabilizers. If $G$ is a group acting on a set $S$, and $s$ is some fixed element of $S$, the \emph{stabilizer} of $s$ is the set of elements which leave it invariant under multiplication proper: $G_s = \{ g \in G | g \cdot s = s \}$. The stabilizer is also a subgroup.
\item Kernel: is the generalization of stabilizer to sets $S$ from points $s$. $g$ must leave \emph{all} points invariant under multiplication.
\end{itemize}

\subsection{Cyclic groups and subgroups}
%==2.3== Cyclic groups and subgroups
\begin{itemize}
\item Cyclic group. A group is cyclic if it is generated by a single element (so taking powers of that element creates all elements of the group). Contrast: Cyclic permutations.
\item Notation: The cyclic group generated by $x$ is denoted by $\langle x \rangle$. The inverse of $x$ generates the same group as $x$ itself: notice $x^{-n} = (x^{-1})^n$ follows from the group axioms.
\item Examples: The rotations in the dihedral group are generated by $r$. The integers are generated by $1$. 
\end{itemize}

\begin{theorem}[Proposition 2]
If $H$ is generated by $x$, then the size of $H$ is equal to the order of $x$. In particular:
\begin{itemize}
\item $x$ of finite order generate $H$ with distinct elements $1, x, \ldots, x^{n-1}$;
\item Otherwise, all elements of $H$ generated by different powers of $x$ are distinct. ($x^a \neq x^b$).
\end{itemize}
\end{theorem}
In pithy notation, the theorem states $H = \langle x \rangle \implies |H| = |x|$.

\begin{lemma}
If $x^m = x^n = 1$, then $x^{\text{gcd}(m,n)} = 1$. 
\end{lemma}
Proof: By an application of the extended euclidean algorithm, which states that the gcd of $m,n$ can always be expressed as a linear combination of $m$ and $n$.

\begin{lemma} Cyclic groups of the same order are isomorphic (so they can be defined in a way that preserves the group multiplication: the map is $\langle x \rangle \to \langle y \rangle, x^k \mapsto y^k$.). If a cyclic group has infinite order, then it is isomorphic to $\mathbb{Z}$ (take $x^k \maps to k$.)
\end{lemma} 
(Proof: is an entire page long somehow.)

\begin{theorem}[Relating $x$ to powers of $x$ (Proposition 5).]
\begin{itemize}
     \item If $x$ has infinite order, then so does any power of $x$ (no doy.)
     \item If $x$ has finite order, then the order of $x^a$ is $\frac{|x|}{gcd(|x|, a)}$
     \item If $a$ divides $n$, the above simplifies to $|x^a| = \frac{|x|}{a}$.
\end{itemize}
\end{theorem}
\begin{proof}
Proof of (2): let $d$ be the gcd of $a$ and $n = |x|$. then by construction, we can decompose  $n = db, a = dc$, where b,c are relatively prime. 
Then $\frac{n}{d} = \frac{db}{d} = b$, so we wish to show $|x^a| = b$. Now $(x^{a})^{b} = x^{ab} = x^{(dc)(b)} = (x^{db})^{c} = (x^{n})^{c} = (1)^c = 1$; so $|x^a|$ divides $b$ (We don't know that the order is $b$ just yet, only that the order is a factor of $b$).
Also, $x^{a|x^a|} = (x^a)^{|x^a|} = 1$, so $n$ divides $a |x^a|$ (by the same reasoning as above). By cancelling out common factors, we conclude that $b$ must divide $|x^a|$.
Since $|x^a|$ divides $b$ and $b$ divides $|x^a|$, $b = |x^a|$.
Note: (3) is a useful special case of (2).
\end{proof}
\begin{theorem}[Proposition 6]
Let $H$ be generated by $x$. Then:
\begin{itemize}
\item If $|x| = \infty$, then $H = \langle x^a \rangle$ iff $a = \pm 1$.
\item If $|x| < \infty$, then $H = \langle x^a \rangle$ iff $\text{gcd}(a,n) = 1$. In particular, the number of generators of $H$ is $\phi(n)$, where $\phi$ is Euler's $\phi-$function. (aside: wat)
\end{itemize}
\end{theorem}
\begin{proof}
(1) (was left as an exercise.) Attempt: Consider the element $x^{a - 1}$. We will show that it is not an element of $\langle x^a \rangle$: first note that $a - 1$ cannot divide $a$: if it did, then $k(a - 1) = a \implies a(k - 1) = 1, but the only elements in the integers which have an integer-valued inverse are $\pm 1$ (the opposite should also be true). Next, by the definition of order, all the elements $x^r$ are distinct. In particular, $x^{a - 1} \neq x^{ka}$ for any integer $k$. Since $x^{ka} = (x^a)^k$, this completes the proof.
(2) By proposition 2, $x^a$ generates a subgroup of $H$ of size $|x^a|$. This subgroup equals all of $H$ iff $|x^a| = |x|$. By proposition 5, this is true iff $\frac{n}{\text{gcd}(a,n)} = n$ iff $\text{gcd}(a,n) = 1$.
$\phi$ (Euler's $\phi$ function) is defined to be the number of integers $a$, $0 \leq a \leq n$ which are relatively prime to $n$. Thus by construction of that function, $\phi(|x|)$ gives the number of generators of $H$.
\end{proof}

Example: Proposition 6 tells precisely which residue classes mod $n$ generate $\frac{\mathbb{Z}}{n \mathbb{Z}}$ (which has order $n$): $\bar{a}$ generates $\frac{\mathbb{Z}}{n \mathbb{Z}}$ iff $\text{gcd}(a,n) = 1$. For instance, $1, 5, 7, 11$ are all generators of $\frac{\mathbb{Z}}{12 \mathbb{Z}}$, and $\phi(12) = 4$.

\begin{theorem}[Theorem 7]
Let $H = \langle x \rangle$ be a cyclic group.
\begin{itemize}
\item Every! subgroup of $H$ is cyclic: if $K \leq H$ ($\leq$ implies $K$ is subgroup), then either $K = \{ 1 \}$ or $K = \langle x^d \rangle$, where $d$ is the smallest possible integer such that $x^d \in K$.
\item If $|H| = \infty$, then for any distinct nonnegative integers $a,b$, $\langle x^a \rangle \neq \langle x^b \rangle$. Furthermore, for every integer $m$, $\langle x^m \rangle = \langle x^{|m|} \rangle$, where $|m| = \text{abs}(m)$. Thus the nontrivial subgroups of $H$ correspond bijectively with the positive integers $1, 2, 3, \ldots$.
\item If $|H| = n < \infty$, then for each positive integer $a$ dividing $n$, there is a unique subgroup of $H$ of order $a$. This subgroup is the cyclic group $\langle x^d \rangle$, where $d = \frac{n}{a}$. Furthermore, for every integer $m$, $\langle x^m \rangle = \langle x^{\text{gcd}(n,m)} \rangle$, so that the subgroups of $H$ correspond bijectively with the positive divisors of $n$. 
\end{itemize}
\end{theorem}
\begin{proof}
% let me know if there are sections you'd like to see.
(3): Assume $|H| = n < \infty$ and that $a | n$. Let $d = \frac{n}{a}$ and apply proposition 5 part 3 to obtain that $\langle x^d \rangle$ is a subgroup of order $a$, showing the existence of a subgroup of order $a$. To show the uniqueness of the subgroup, suppose $K$ is any subgroup of $H$ with order $a$. By part (1) we have $K = \langle x^b \rangle$, where $b$ is the smallest positive integer such that $x^b \in K$. By proposition 5, $\frac{n}{d} = a = |K| = |x^b| = \frac{n}{\text{gcd}(n,b)}, so $d = \text{gcd}(n,b)$. In particular $d |b$. Since $b$ is a multiple of $d$, $x^b \in \langle x^d \rangle$, and hence $K = \langle x^d \rangle \leq \langle x^d \rangle$. Since the order of the group generated by $x^d$ is $a = K$, we have $K = \langle x^d \rangle$.
The final assertion made in (3) follows from the observation that $\langle x^m \rangle$ is a subgroup of $\langle x^{\text{gcd}(n,m)} \rangle$. From proposition 5 part 2 and proposition 2, the subgroups must have the same order. Since $\text{gcd}(n,m)$ is certainly a divisor of $n$, this shows that every subgroup of $H$ arises from a divisor of $n$, completing the proof.
\end{proof}


\question (26) Let $Z_n$ be a cyclic group of order $n$, and let $k$ be an integer which is relatively prime to $n$. Prove that the map $x \mapsto x^k$ is surjective. Then use Lagrange's Theorem to prove that the same is true for \emph{any} finite group of order $n$. Note: For such $k$ each element has a $k$th root in $G$. It follows from Cauchy's theorem (section 3.2) that if $k$ is not relatively prime to the order of $G$, then the map $x \mapsto x^k$ is not surjective.

\subsection{2.4 Subgroups generated by subsets of a group}

Fun fact: The cyclic group generated by $x$ is the \emph{smallest subgroup} containing $x$. In general we can also define the smallest subgroup containing some set $K$. This is a common theme in mathematics: Given some object $G$, and some subset of that object $A$, is there a unique minimal object containing $A$, in the sense that any other object containing $A$ must also contain that object? How are elements of this subobject computed?
Warning: Similar sub-objects will be defined for other algebraic objects later on, and we won't take as much care with them. Without further ado, let's define and prove properties of subgroups generated by subsets.

\begin{theorem}[Proposition 8]
If $\mathcal{A}$ is any nonempty collection of subgroups of $G$, then the intersection of all members of $\mathcal{A}$ is also a subgroup.
\end{theorem}
\begin{proof}
The intersection is nonempty because all the subgroups contain the identity.
The closure properties follow from the closure properties holding in all of the subgroups in $\mathcal{A}$.
\end{proof}

Definition. Let $A$ be a subset of the group $G$. The \emph{subgroup generated by $A$} is 
\begin{equation}
\langle A \rangle = \bigcap_{A \subseteq H, H \leq G} H$.
\end{equation}
By proposition 8, $\langle A \rangle$ is indeed a subgroup.

When $A$ is the finite set $\{ a_1, \ldots, a_n \}$, we write $\langle a_1, \ldots, a_n \rangle$ instead of (the other notation). If $A, B$ are two subsets, we write $\langle A, B \rangle$ instead of $\langle A \cup B \rangle$.

Getting a more constructive definition: Let $\bar{A} = \{ a_1^{\epsilon_1} a_2^{\epsilon2} \cdots a_n^{\epsilon_n} | n \in \mathbb{Z}, n \geq 0 \text{ and } a_i \in A, e_i = \pm 1 \text{ for each } i \}$. If $A$ is empty, let $\bar{A} = \{1\}$. This is the `closure` of $A$ under the group operation (and the process of taking inverses). The elements of the closure (all finite products) are also called \emph{words}. Note that the $a_i$ need not be distinct (so it is possible to write $a^2$ -- write it as $aa$). Also note $A$ need not be finite (or even countable) for this definition to make sense.

\begin{theorem}[Proposition 9]
$\bar{A} = \langle A \rangle$.
\end{theorem}
\begin{proof}
% later....
\end{proof}

In light of the above, let's always use the notation $\langle A \rangle$ from now on. An alternate way of defining $\langle A \rangle$ is $\{ a_1^{\alpha_1} \cdots a_n^{\alpha_n} | \text{ for each } i, a_i \in A, \alpha_i \in \mathbb{Z}, a_i \neq a_{i + 1} \text{ and } n \in \mathbb{Z}^{+} \}$ (so uh, use power notation instead?)

If $G$ is abelian we can make further simplifications. 
If we also assume each $a_i$ has finite order $d_i$, then $| \langle A \rangle | \leq \prod_{i} d_i$.

Also, it might be the case that $a^{\alpha} b^{\beta} \neq (...)$ even when (lhs) does not equal (rhs).

If $G$ is non-abelian, things are more complicated. (then some silly examples follow. To note: in the general linear group, finite subsets can generate infinite subgroups. Oy!)
% request examples here if you would like them.

\subsection{2.5 - Lattices of subgroups}
This section has a lot of diagrams. They basically look like prime factorization tree diagrams from grade school, except the ordering is groups and subgroups instead of numbers and their factors.
Notably, the Klein-4 group (no, not that one) has the shape of a diamond.
Secretly these are also (CS) graphs. Doy!

Oh, here's a formal definition: Lattices are partial orders of a group and its subgroups, where the ordering is given by inclusion.
The unique smallest subgroup which contains the subgroups $H, K$ is called the \emph{join} of $H, K$ and is denoted by $\langle H, K \rangle$.

\subsection{3.1 - Quotient Groups}
% yay quotient groups

the lattice of subgroups for a quotient of G is reflected at the ``top'' (in a precise sense) of the lattice for G whereas the lattice for a subgroup of G occurs naturally at the bottom.
The study of quotient groups is equivalent to the study of homomorphisms of $G$ , as we shall see.
Suppose phi is a homomorphism from G to H, and recall that the fibers of phi are the sets of elements of G which project to single elements of H.
The group operation of H provides a way to multiply two elememts in the image pf phi, which suggest a natural multiplication for the fibers lying above the two points. Thus the set of fibers becomes a group: If $X_a, X_b$ are the fibers above $a, b \in H$ resp., then we define $X_a \dot X_b = X_{ab}$.
Roughly speaking, the group $G$ is partitioned into pieces (the fibers) and these pieces themselves have the structure of a group, called a quotient group of G.
Since the multiplication of fibers is defined from multiplication in $H$, by construction the quotient group is isomorphic to the image of $G$ under the homomorphism phi. (The fiber X_a is identified with its image a in H.)

(an example with modular arithmetic -- I'm underselling it but w/e will type later)

Definition. If phi is a homomorphism from G to H, the kernel of phi is the set $\{ g \in G | \phi(g) = 1 \}$ and will be denoted by $\text{ker }\phi$ (here 1 is the identity of $H$).

\begin{theorem}[Proposition 1]
Let $G$ and $H$ by groups and let \phi : G \to H be a homomorphism.
\begin{enumerate}
\item $\phi(1_G) = 1_H$,
\item $\phi(g^{-1}) = \phi(g)^{-1}$
\item $\phi(g^n) = \phi(g)^n$
\item $\text{ker}(\phi)$ is a subgroup of $G$
\item $\text{im}(\phi)$, the image of $G$ under $\phi$, is a subgroup of $H$.
\end{enumerate}
\end{theorem}
\begin{proof}
(1) Since $\phi(1_G) = \phi(1_G \cdot 1_G) = \phi(1_G) \cdot \phi(1_G), the cancellation laws show that (1) holds. 
(2) 1_H = \phi(g \cdot g^{-1}) = phi(g^{-1}) \cdot \phi(g). By uniqueness of inverse, $\phi(g^{-1}) = \phi(g)$. The first equality follows from (1).
(3) Even the authors don't want to do this one. Note part (2) implies the equality for negative $n$ also.
(4) Use the textbook subgroup criterion: Suppose $x,y \in \text{ker}(\phi)$ and check that $xy^{-1}$ is as well: $\phi(xy^{-1}) = \phi(x)\phi(y^{-1}) = 1 \cdot 1^{-1} = 1$.
(5) Careful: The image resides in $H$, not $G$. Check, for $x,y \in H$: $xy^{-1} = \phi(q) \phi(r)^{-1} = \phi(qr^{-1})$. Since $G$ is a group, we're done. (Also be careful to check that the image is nonempty. Use part 1 for this.)
\end{proof}

% Alternate pf of (1)
% (1) $1_H \cdot \phi(x) = \phi(x) = \phi(1_G \cdot x) = \phi(1_G) \cdot \phi(x).
% Then use the cancellation property and uniqueness of inverse. (don't worry too much about fibers; these are essentially just values.)

Definition. The quotient group $G / K$ (read: G modulo K or simply G mod K), is the group whose elements are the fibers of $\phi$ with group operation defined above.

The notation emphsizes the fact that the kernel is a single element in the quotient group and that the other elements are just the ``translates'' of the kernel. Hence we may think of $G/K$ as being obtained by ``dividing out'' K from G. This is why $G / K$ is called a ``quotient'' group.

Note: Instead of defining phi explicitly, we can also define multiplication of fibers directly in terms of representatives from the fibers.

\begin{theorem}[Proposition 2]
Let $\phi$ be a homomorphism (...) and $X$ be the fiber above $a$. Then
\begin{enumerate}
\item For u in X, $X = \{ uk | k \in K \}$
\item For u in X, $X = \{ ku | k \in K \}$.
(note that these are respectively, the left and right cosets of $K$ wrt $u$.)
\end{enumerate}
\end{theorem}
\begin{proof}
(1) : Certainly $uk \in X$ given $k \in K$. On the other hand, suppose $b$ belongs to $X$. Then $1_H = a a^{-1} = \phi(b) \phi(u)^{-1} = \phi(b \cdot u^{-1}), so $b u^{-1}$ lies in $K$. But this means precisely that $b u^{-1} = k \implies b = uk$. (2) is similar.
\end{proof}

Definition (actually). For any $N \leq G$ and $g \in G$, let $gN = \{ gn | n \in N \}, Ng = \{ ng | n \in N \}$. These are the \emph{left and right cosets} of $N$ with respect to $g$, respectively. \emph{Any element of a coset is called a representative for the coset.}

If $N$ is the kernel of a homomorphism, and $g_1$ is any representative for $gN$, then $g_1 N = gN$. In fact, this is valid for arbitrary subgroups $N$, which explains the terminology of a representative.

Using additive notation, we can also write $g + N, N + g$ for the left and right cosets of $N$ wrt $g$. Recall: The right cosets of $N$ in $G$ are precisely the orbits of $N$ acting on $G$ by left multiplication. (quoting terminology from 1.7.18 -- the orbit of $x$ in a set $A$ is the equivalence class of $x$ under the usual equivalence relation $x \sym y \iff x = ny, n \in N$. So.. this is kind of abuse of terminology by the textbook?)

Thus we just showed that the \emph{fibers of a homomorphism are the left/right cosets of the kernel}.

\begin{theorem}[Theorem 3]
Let $K$ be the kernel of any homomorphism from $G$ to some other group. Then the left cosets of $K$ with group operation defined by $uK \circ vK = (uv)K$ forms the quotient group $G / K$.
This group operation is well-defined: Given the two cosets $uK, vK$, if $u_1, v_1$ are elements of those respective cosets, then $u_1 v_1 \in (uv)K$. In other words, the group operation \emph{does not depend} on the representatives we choose for the cosets.
\end{theorem}
\begin{proof}
Let $X,Y \in G / K$, and let $Z = XY$ in $G/K$, so that by proposition 2, $X$, $Y$ and $Z$ are left cosets of $K$. By assumption, $K$ is the kernel of some homomorphism $\phi: G \to H$. Thus $X = \phi^{-1}(a), Y = \phi^{-1}(b)$ for $a,b \in H$ (so $X = aK, Y = bK$). Then $Z = \phi^{-1}(ab)$ by definition of the group operation.

Now let $u, v$ be arbitrary representatives of $X, Y$ resp. We must show that $uv \in Z$. 

\begin{align*}
uv \in Z \iff uv \in \phi^{-1}(ab) \\
\iff \phi(uv) = ab 
\iff \phi(u)\phi(v) = ab 
\end{align*}

The last line is true, so $uv \in Z$, and in particular (again by proposition 2), $Z = uvK$. This proves that the product of $X$ with $Y$ is the coset $uvK$, for any choice of representative $u,v$. The last statement of the theorem follows from the fact that $uK = Ku, vK = Kv$ for all $u,v \in G$.
\end{proof}

As a concrete example of ``reducing a group to only its cosets'', notice that in $\frac{\mathbb{Z}}{5 \mathbb{Z}}, we don't distiguish between 6 and 11 -- both are 1 mod 5. (The homomorphism takes elements to their mod 5 equivalents.)

\begin{theorem}[Proposition 4]
Let $N$ be any subgroup of $G$. Then the set of left cosets of $N$ in $G$ form a partition of $G$, $uN = vN$ iff $uv^{-1} \in N$ iff $u, v$ are representatives of the same coset.
\end{theorem}
\begin{proof}
Here is a sketch: Show that distinct left cosets have empty intersection by assuming otherwise and obtaining a contradiction.
The second part of the theorem follows from the first after symbolic manipulations. The third part is definitions.
\end{proof}

\begin{theorem}[Proposition 5]
Let $N$ be a subgroup of $G$.
\begin{enumerate}
\item The group operation on left cosets is well-defined iff $gng^{-1} \in N$ for all $g \in G, n \in N$. (This can be rephrased as saying that the normalizer of $N$ is all of $G$.)
\item When the group operation on left cosets is well-defined, the left cosets of $N$ for a group.
\end{enumerate}
\end{theorem}
\begin{proof}
Assume first that the operation is well-defined. Fix $g, n$. Notice that the coset $nN = 1N$, so that $(1g^{-1})N = (ng^{-1})N$ (so that g^{-1} and ng^{-1} are also representatives for the same coset). Thus g^{-1}n' = ng^{-1}, for some $n' \in N$. Multiplying both sides on the left by $g$ shows that the conjugate $g n g^{-1}$ belongs to $N$, as desired.
Conversely, assme that $g n g^{-1} \in N$ for all $g, n$. To show that the operation is well-defined, let $u, u_1 \in uN, v, v_1 \in vN$. Then (by previous theorems) we can write $u_1 = n, v_1 = vm$ for $n,m \in N$. We must show that $u_1 v_1 \in (uv)N$: $u_1v_1 = (un)(vm) = uvv^{-1}nvm = uvn'm = (uv)n''$ for $n'' \in N$, as desired.
(2) : Proof in textbook, should be straightforward.
\end{proof}

Definitions (which are equivalent to saying that normal subgroups have left and right cosets equal): $gng^{-1}$ is called the conjugate of $n \in N$ by $H$. The set $gNg^{-1} = \{ gng^{-1} | n \in N \}$ is called the conjgate of $N$ by $g$. $g$ is said to normalize $N$ if $gNg^{-1} = N$. (so it normalizes all the $n$). $N$ is normal if every element of $G$ normalizes $N$. If $N$ is a normal subgroup of $G$, we write $N \trianglelefteq G$.

The following theorem summarizes these results.
\begin{theorem}[Theorem 6]
The following are equivalent:
\begin{enumerate}
\item $N \trianglelefteq G$
\item The normalizer of $N$ in $G$ is equal to $G$ (Notation: $N_G(N) = G$)
\item $gN = Ng$ for all $g$
\item The quotient group is \emph{actually} a group
\item all the $g$ normalize $N$ (oy. Notation: $gNg^{-1} \subseteq N$ for all $g \in G$ (why $\subseteq$ now? weird.)
\end{enumerate}
\end{theorem}

As you might have guessed by now, $N$ is normal iff it is the kernel of some homomorphism (Proposition 7). We can explicitly construct it: $\pi: g \to G/N, \pi(g) = gN$ for all $g \in G$. (proof omitted.)

Definition. The homomorphism $\pi$ is called the natural projection (aside: `natural' is actually a defined term in category theory; it loosely translates to being `coordinate-free') of $G$ onto $G/N$. If $\bar{H} \leq G/N$, then we define the \emph{complete preimage} of $\bar{H}$ in $G$ to be the preimage of $\bar{H}$ under the natural projection homomorphism.

Note: The complete preimage is a subgroup of $G$ which contains $N$.
Note 2: We can in some sense think of the normalizer of a subgroup $N$ to measure `how close' $N$ is to being a normal subgroup. However, keep in mind that normality is an embedding property, not an intrinsic one: it depends on the ambient space which $N$ is embedded in. 

Conclusion: The study of homomorphic images of $G$ is \emph{equivalent} to the study of quotient groups of $G$.

Some examples: Abelian groups have all subgroups normal. Quotient groups of a cyclic group are always cyclic. (...)

% Finally! Done this section. Exercises follow.
\question (10) Let $\phi : \frac{\mathbb{Z}}{8 \mathbb{Z}} \to \frac{\mathbb{Z}}{4 \mathbb{Z}}$ by $\phi(\bar{a}) = \bar{a}$. Show that this is a well-defined, srjective homomorphism, and describe its fibers and kernel explicitly (Note: showing that $\phi$ is well-defined involves the fact that $\bar{a}$ has a different meaning in the domain and range of $\phi$.)

The order of the quotient group of a (finite) group has an easy formula (following Lagrange's theorem): it is 
\begin{equation}
|G/N| = \frac{|G|}{|N|}.
\end{equation}
This is a very important result, one of the most important. Make sure to triple star it in your notes. (oy.)

\begin{theorem}[Theorem 8. Lagrange's theorem]
If $G$ is a finite group and $H$ is a subgroup of $G$, then the order of $H$ divides the order of $G$, and the number of left cosets of $H$ in $G$ is $\frac{G}{H}$.
\end{theorem}
\begin{proof}
Let $n = |H|$. Let $k$ be the number of left cosets (`|G/H|'). By proposition 4, the left cosets of $H$ partition $G$ (since $H$ is a subgroup). Also, the map $H \to gH, h \mapsto gh$ is a surjection. The left cancellation law implies this map is also injective, since $gh_1 = gh_2$ implies $h_1 = h_2$. Thus $H$ and $gH$ have the same order.
Since $G$ is partitioned into $k$ disjoint subsets of order $n$, $|G| = kn$. This completes the proof.
\end{proof}

Definition. If $G$ is a (possibly infinite) group and $H \leq G$, the number of left cosets of $H$ in $G$ is called the \emph{index} of $H$ in $G$. It is denoted by $|G : H|$.

Easy consequences of Lagrange's:
(Corollary 9.) If $G$ is finite, $x \in G$, then $|x| divides |G|$, and so $x^{|G|} = 1$ for all $x$. Proof: $\langle x \rangle$ is a subgroup of $G$ and has the size equal to the order of $x$. Apply Lagrange.
(Corollary 10.) If $G$ is a group of prime order $p$, then $G$ is cyclic, and $G \cong Z_p$. Proof: Let $x \in G, x \neq 1$. So the size of the cyclic group generated by $x$ is greater than 1, and it must divide the order of $G$, by the above. But since $|G|$ is prime, the orders must actually be equal, which implies that $G = \langle x \rangle$ (uh how). Theorem 2.4 completes the proof.

(some examples -- probably important, will do later)
(more examples, for non-normal subgroups)

Note: The converse to Lagrange is not true, ie. if $G$ is finite and $n$ divides $|G|$, it's not true that $G$ has a subgroup of order $n$.

\begin{theorem}[Theorem 11, Cauchy's Theorem]
If $G$ is a finite group, and $p$ is a prime divisor of $|G|$, $G$ must have an element of order $p$.
\end{theorem}
(The proof: In the exercises)

\begin{theorem}[Theorem 12, Sylow]
If $G$ has a finite group of order $p^{\alpha}m$, $p$ prime, then $G$ has a subgroup of order $p^{\alpha}$.
\end{theorem}
This is the `strongest converse to Lagrange that we have'. Proof: Next chapter. Yay.

Definition. Let $H,K$ be subgroups. Let $HK = \{ hk, h \in H, k \in K \}$.

\begin{theorem}[Proposition 13]
$|HK| = \frac{|H||K|}{|H \cap K|}$ (huh. conditional probability product rule?)
\end{theorem}
\begin{proof}
%later...
\end{proof}

\begin{theorem}[Proposition 14]
If $H$, $K$ are subgroups, $HK$ is a subgroup iff $HK = KH$.
\end{theorem}
\begin{proof}
% seems simple
\end{proof}

(Corollary 15.) Let $H,K$ be subgroups of $G$. $HK$ is a subgroup of $G$ if $H \leq N_G(K)$. 

Definition. If $A$ is a subset of $N_G(K)$ (or $C_G(K)$) we say that $A$ normalizes [centralizes] $K$.

(then some discussion)

\question (11)

%==Week 4 Handout==
\question (4)
\question (5)

%==Worksheet: Cyclic Groups==

%==Handout: Burnside's Lemma==
\end{questions}

\section{Weeks 5 and 6}


\end{document}  
